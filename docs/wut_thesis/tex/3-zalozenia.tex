\newpage
\section{Założenia}
Treść sekcji zawiera zestawienie dwóch konkurencyjnych architektur wykorzystywanych do
projektowania systemów informatycznych: monolitycznej oraz opartej na mikrousługach.
Podrozdziały 4.1.1 - 4.1.4 przedstawiają zagadnienia, na które należy zwrócić
uwagę, przygotowując system utworzony z wielu modułów usługowych. W podrozdziale 4.1.5
spisano wszystkie moduły, które składają się na zaprojektowany w ramach niniejszej pracy 
system.

W tej sekcji przedstawiono główne założenia dotyczące rozwoju pracy, a także
przeprowadzono przegląd prac naukowych estymujących optymalną wartość temperatury
oraz natężęnia światła w pomieszczeniach biurowych.

Funkcjonalność i architektura systemu została zaprojektowana w oparciu o kilka istotnych 
założeń:

\begin{itemize}
    \item Szczególny nacisk jest położony na łatwość wdrożenia - system powinien być 
    gotowy do wdrożenia na środowisko 
    chmurowe. Organizacja zainteresowana uruchomieniem systemu dla swoich potrzeb 
    może wybrać opcję, w której dostarczane są obrazy odpowiednich mikroserwisów oraz 
    skrypty konfigurujące środowisko. Takie rozwiązanie mogłoby być ofertą skierowaną 
    do banków, które chcą zminimalizować ruch zewnętrzny. Może także skorzystać z 
    opcji, w której system jest hostowany na serwerach firmy będącej autorem 
    oprogramowania
    \item System składa się z czujników zbierających pomiary, które następnie przesyłane 
    są do mikroserwisów, które je przetwarzają. Do pomiaru zalicza się aktualna 
    temperatura oraz natężenie światła
    \item System oparty jest na regułach określających oczekiwaną wartość powyższych 
    parametrów w danej chwili czasu. Po otrzymaniu każdego z pomiarów porównywane są 
    wartości oczekiwane z rzeczywistymi i na tej podstawie przygotowywany jest wynik. 
    Domyślnie istnieje reguła podstawowa, gdzie oczekiwana temperatura wynosi 24.5
    \degree C, natomiast oczekiwane natężenie światła wynosi 550 lx. 
    Więcej informacji odnośnie ustalonych wartości zawarto 
    w tabeli \ref{tab:optymalna-temperatura} oraz tabeli \ref{tab:optymalne-natezenie}.
    \item System przewiduje dwie role użytkowników: pracowników danej 
    organizacji, którzy mogą tworzyć własne reguły dla pomieszczeń do nich 
    przypisanych, oraz administratorów organizacji, którzy posiadają wszystkie 
    uprawnienia przypisane pracownikom, a ponadto możliwość zarządzania informacjami 
    dotyczącymi organizacji, budynków, pomieszczeń i czujników
    \item System jest rozwijany przy wykorzystaniu platformy .NET w wersji 5.0
\end{itemize}

Tabela \ref{tab:optymalna-temperatura} pokazuje porównanie wyników z różnych artykułów traktujących 
o optymalnej temperaturze w pomieszczeniach:

\begin{xltabular}{1\textwidth}
    { |c|c| }
    \caption{Porównanie wyników badań estymujących optymalną temperaturę} \label{tab:optymalna-temperatura} \\
     \hline
     Badanie & Optymalna temperatura \\ 
     \hline
     \cite{Lan2012} & 23.5\degree C - 25.5\degree C \\ 
     \hline
     \cite{dai2014} & 23.0\degree C - 26.5\degree C \\ 
     \hline
     \cite{hedge2005} & 24.0\degree C - 25.0\degree C \\ 
     \hline
\end{xltabular}

W oparciu o powyższe badania wyliczono optymalną temperaturę jako średnią z uzyskanych
rezultatów. Jej wartość wynosi 24.5\degree C.
Natomiast w tabeli \ref{tab:optymalne-natezenie} porównywane są rezultaty badań nad optymalnym 
poziomem natężenia światła, który skutkował najlepszą efektywnością pracowników:

\begin{xltabular}{1\textwidth}
    { |c|c| }
    \caption{Porównanie wyników badań estymujących optymalne natężenie światła} \label{tab:optymalne-natezenie} \\
     \hline
     Badanie & Optymalne natężenie światła \\ 
     \hline
     \parencite{chinchiuan2014} & 500 lx \\ 
     \hline
     \parencite{liu2017} & 600 lx \\ 
     \hline
\end{xltabular}

W oparciu o powyższe badania wyliczono optymalne natężenie światła jako średnią z uzyskanych
rezultatów. Jej wartość wynosi 550 lx.