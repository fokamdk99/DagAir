\newpage
\section{Pomiary}

W ramach tej pracy inżynierskiej przygotowano zestaw pomiarowy, który 
w regularnych odstępach czasu bada aktualną wartość temperatury oraz
natężenia światła. Zestaw został złożony z następujących elementów:

\begin{itemize} % lista nienumerowana
    \item Moduł WEMOS D1 Uno R3 ESP8266 WIFI
    \item Fotorezystor LDR GL5528
    \item Czujnik temperatury i wilgotności DHT11
    \item Rezystory 1 kOhm
\end{itemize}

Moduł WEMOS jest mikrokontrolerem oparty o kontroler ATmega328P. Posiada
11 portów wejścia-wyjścia pozwalających na dołączenie zewnętrznych 
urządzeń. Płytka została fabrycznie wyposażona w moduł WIFI ESP8266. Za
jego pomocą mogą zostać wysłane pomiary.

Fotorezystor służy do pomiaru natężenia światła. Został wykonany 
z półprzewodników, które w temperaturze działania nie mają elektronów 
w paśmie przewodnictwa. Padające na półprzewodnik fotony o energii 
większej od przerwy energetycznej przemieszczają elektrony z pasma 
walencyjnego do pasma przewodnictwa, w wyniku którego powstają pary 
dziura-elektron. Zjawisko nazywane jest efektem fotoelektrycznym 
wewnętrznym.

Czujnik temperatury i wilgotności DHT11 bada aktualne wartości tych
dwóch parametrów. Mierzony zakres temperatury to -20\degree  - +60\degree C.
Jego rozdzielczość wynosi 0.1\degree C, a dokładność 2\degree C.

\subsection{Oprogramowanie mikrokontrolera}

Do rozwoju oprogramowania do mikrokontrolera WEMOSC wykorzystano narzędzie 
Arduino IDE, będące rozbudowanym edytorem pozwalającym na kompilację,
przesyłanie plików wykonywalnych na płytkę przy pomocy kabla USB oraz
debugowanie i podgląd logów produkowanych przez mikrokontroler w czasie 
rzeczywistym. Do utworzenia kodu został wykorzystany język C.

Zadaniem mikrokontrolera było:

\begin{itemize} % lista nienumerowana
    \item Zebranie aktualnych pomiarów temperatury i natężenia światła
    \item przygotowanie wiadomości z uzyskanymi pomiarami
    \item wysłanie wiadomości przy pomocy protokołu MQTT przez moduł WIFI na 
    kolejkę brokera wiadomości RabbitMQ
\end{itemize}

Poniżej został przedstawiony przygotowany kod.

\begin{lstlisting}
    #include "rabbitmq_handler.h"

    #define WIFI_SSID "Tech_D0054234"
    #define WIFI_PASS "PASSWORD"
     
    void setup() {
      Serial.begin(115200);
      dht.begin();
      Serial.println();
     
      WiFi.mode(WIFI_STA);
      WiFi.begin(WIFI_SSID, WIFI_PASS);
     
      while (WiFi.status() != WL_CONNECTED)
      {
        delay(100);
      }
      
      client.setServer(RABBITMQ_BROKER, RABBITMQ_PORT);
      client.setCallback(callback);
    }
     
    void loop() {
      if ( !client.connected() ) {
        reconnect();
      }
      
      float temperature = readTemperature(&dht);
      int illuminance = analogRead(ILLUMINANCEPIN);
      publish_measurements(temperature, illuminance, false);
      
      delay(60000);
    
      client.loop();
    }
\end{lstlisting}

Każdy z programów wgrywanych na płytkę powinien zawierać dwie główne funkcje:

\begin{itemize}
    \item void setup() - wewnątrz metody definiowane są obiekty oraz zmienne 
    potrzebne przez cały czas działania mikrokontrolera
    \item void loop() - metoda wywoływana jako druga w kolejności po setup(),
    jest powtarzana przez resztę cyklu działania mikrokontrolera 
\end{itemize}

Na początku definiowana jest szyna, za pomocą której przesyłane są dane do 
modułu WIFI. Moduł wspiera prędkość transmisji na poziomie 115200 bitów na 
sekundę (ang. \textit{baud rate}). Następnie tworzony jest klient, którego zadaniem
jest wysyłanie pomiarów na kolejkę brokera wiadomości. 

Wewnątrz metody loop() co 60 sekund zbierane są pomiary, po czym publikowane
na kolejkę. Pomiary wykonywane są przy użyciu poniższego kodu.

\begin{lstlisting}
    #include "DHT.h"

    #define DHTPIN 4
    #define DHTTYPE DHT11
    
    int ILLUMINANCEPIN = A0;
    
    DHT dht(DHTPIN, DHTTYPE);
    
    float readTemperature(DHT *dht){
      float t = dht->readTemperature();
     
      if (isnan(t))
      {
        Serial.println(
            "Error while reading current 
            temperature value");
      }
      
      return t;
    }
\end{lstlisting}

Główną rolę pełni obiekt dht, za pomocą którego można komunikować się
z czujnikiem. Zdefiniowany jest typ czujnika przy użyciu makra DHTTYPE.
Numer pinu, do którego został wpięty kabel łączący płytkę z czujnikiem,
został oznaczony pzy użyciu makra DHTPIN. Pomiar temperatury odbywa się
przez wywołanie funkcji readTemperature(). 

Pomiar z fotorezystora można odczytać przy użyciu komendy:  

\begin{lstlisting}
    analogRead(ILLUMINANCEPIN)
\end{lstlisting}

gdzie ILLUMINANCEPIN oznacza numer pinu, do 
którego wpięty jest kabel łączący płytkę z fotorezystorem.

Poniższy kod zawiera szczegóły implementacyjne dotyczące 
wysyłania wiadomości na kolejkę.



Warto wyjaśnić wartości zmiennych związanych z brokerem wiadomości:
\begin{lstlisting}
    const char* RABBITMQ_BROKER = "192.168.0.12";
    int        RABBITMQ_PORT     = 1883;
    const char* RABBITMQ_TOPIC  = "room_measurements";
    const char* RABBITMQ_SUBSCRIPTION  
    = "request_measurement";
    const char* RABBITMQ_USER = "guest";
    const char* RABBITMQ_PASSWORD = "guest";
    const char* RABBITMQ_SENSOR_ID = "968376";
\end{lstlisting}

\begin{itemize}
    \item RABBITMQ\_BROKER: adres IP serwera, na którym uruchomiony jest broker wiadomości
    \item RABBITMQ\_PORT: numer portu serwera, na którym nasłuchuje broker
    \item RABBITMQ\_TOPIC: temat, na który wysyłane są wiadomości z mikrokontrolera
    na kolejkę
    \item RABBITMQ\_SUBSCRIPTION: temat, na który nasłuchuje mikrokontroler
    \item RABBITMQ\_USER: nazwa użytkownika, za pomocą którego płytka jest uwierzytelniana
    \item RABBITMQ\_PASSWORD: hasło dla wykorzystywanego użytkownika
\end{itemize}

Wiadomość do brokera jest wysyłana w metodzie send\_measurements:

\begin{lstlisting}
    void send_measurements(
        float temperature, int illuminance){
      char temperatureChar[64];
      int ret = snprintf(
          temperatureChar, sizeof temperatureChar, 
          "%f", temperature);
      if (ret < 0) {
          return;
      }
      if (ret >= sizeof temperatureChar) {
           return;
      }
      char illuminanceChar[64];
      ret = snprintf(
          illuminanceChar, sizeof illuminanceChar, 
          "%d", illuminance);
      if (ret < 0) {
          return;
      }
      if (ret >= sizeof illuminanceChar) {
           return;
      }
    
      char* measurement = (char*)malloc(256+1+3); 
      //4*64 + 3 semicolons + EOF
      strcpy(measurement, temperatureChar);
      strcat(measurement, ";");
      strcat(measurement, illuminanceChar);
      strcat(measurement, ";");
      strcat(measurement, RABBITMQ_SENSOR_ID);
    
      time_t t = time(NULL);
      struct tm tm = *localtime(&t);
    
      client.publish(RABBITMQ_TOPIC, measurement);
    
      free(measurement);
    }
\end{lstlisting}

Mikrokontroler nasłuchuje na wiadomości o temacie \\RABBITMQ\_SUBSCRIPTION:

\begin{lstlisting}
    void callback(
        char* topic, byte* payload, unsigned int length) {
      char* sensorId = (char*)payload;
    
      String messageTemp;
      
      for (int i = 0; i < length; i++) {
        Serial.print((char)payload[i]);
        messageTemp += (char)payload[i];
      }
    
      if(strcmp(topic, RABBITMQ_SUBSCRIPTION) 
      == 0 && strcmp(sensorId, RABBITMQ_SENSOR_ID)){
        float temperature = readTemperature(&dht);
        int illuminance = analogRead(ILLUMINANCEPIN);
        publish_measurements(
            temperature, humidity, illuminance, true);
      }
    }
\end{lstlisting}

Serwis SSDS może zażądać wysłania aktualnych pomiarów poprzez wysłanie do brokera 
wiadomości o tym samym temacie.