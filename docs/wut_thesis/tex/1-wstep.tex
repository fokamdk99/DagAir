\newpage % Rozdziały zaczynamy od nowej strony.
\section{Wstęp}

Zgodnie z wynikami badań opublikowanymi przez Institute for market transformation z 
2015 roku około 40\% całkowitej konsumpcji energii w Stanach Zjednoczonych przypada na zasilanie budynków 
\cite{Imt.org2015}. Przekłada się to w ciągu roku na wydatek rzędu 450 miliardów 
dolarów. Najsłabiej zagospodarowane budowle zużywają od trzech do siedmiu razy więcej 
energii od tych najbardziej oszczędnych. Istnieje zatem potrzeba przygotowania 
i wdrożenia rozwiązań, które zoptymalizowały by wykorzystanie zasobów energetycznych. 

Utworzenie systemu miało przyczynić się do osiągnięcia dwóch głównych 
celów, stawianych od początku jego przygotowywania:

\begin{enumerate}
    \item Poprawa warunków pracy - badania \cite{oseland2012} dowodzą, że 
    warunki panujące w pomieszczeniach do pracy mają wpływ na efektywność pracowników. 
    Utrzymanie ich na optymalnym poziomie może spowodować wzrost wydajności do 2.5\%
    \item Redukcja zużywanej energii - w praktyce często zdarza się, że po zakończeniu 
    pracy zostawiane są włączone światła na całą noc. Innym przykładem może być 
    sytuacja, w której pomieszczenie jest ogrzewane, mimo iż nikt z niego nie korzysta. 
    Przy wsparciu aplikacji będzie możliwe zapobieganie takim wydarzeniom, co w 
    konsekwencji ograniczy zużycie energii
\end{enumerate}

W odpowiedzi na przedstawione problemy zaprojektowano system do 
zarządzania warunkami technicznymi w pomieszczeniach biurowych oparty na architekturze 
mikrousługowej. Produkt ma na celu poprawę oraz kontrolę zgodności z zadanymi 
wartościami warunków panujących w pomieszczeniach 
przeznaczonych do pracy przy jednoczesnym ograniczeniu zużycia energii. 
Wybrane parametry przeznaczone do optymalizacji to temperatura oraz natężenie światła.

Implementacja projektu przewiduje umieszczenie w pomieszczeniach odpowiedniego 
rodzaju czujników, które będą na bieżąco monitorować stan danej przestrzeni. 
Zintegrowany z czujnikami system informacyjny powinien odczytywać przesyłane 
pomiary, a następnie je interpretować. Wynik interpretacji powinien być widoczny dla 
zainteresowanych osób. W wiadomości będą znajdować się informacje dotyczące 
działań, które należy podjąć, aby umożliwić ustalenie się badanych parametrów na 
właściwym poziomie.

Rozwiązanie było przygotowywane z myślą o jak najprostszym etapie wdrażania. Proces
instalacji systemu w danym biurze powinien składać się z trzech kroków:

\begin{itemize}
    \item Utworzenie nowej organizacji w bazie danych systemu
    \item Umieszczenie czujników w każdym z pomieszczeń, które powinno być monitorowane.
    Po podłączeniu do prądu czujniki powinny przesyłać pomiary
    \item Dodanie umieszczonych urządzeń do listy urządzeń w ramach danej organizacji
\end{itemize}

Procedura instalacji powinna przebiegać szybko, a po jej zakończeniu wszyscy użytkownicy
posiadający odpowiednie uprawnienia powinni otrzymywać wyniki pomiarów oraz rekomendacje
dotyczące dalszych działań.

Dedykowana mikrousługa nasłuchuje na przychodzące pomiary, po czym przesyła
je dalej w celu analizy. W wyniku procesu odpowiednia mikrousługa przetwarzania 
generuje odpowiedź, która zawiera opis aktualnego stanu danego pomieszczenia. Zależnie
od warunków odpowiedź może przyjmować różną postać, np.:

\begin{itemize}
    \item 'Aktualne warunki są zgodne z oczekiwaniami.' - jeśli wartości mierzonych 
    parametrów zgadzają się z ustaleniami
    \item 'Zbyt wysoka temperatura. Rozważ uchylenie okna.' - jeśli pomiar temperatury
    wykracza ponad dozwoloną górną granicę
    \item 'Jest zbyt ciemno. Rozważ włączenie światła.' - gdy pomiar natężenia światła
    wypada poniżej ustalonego dolnego poziomu
\end{itemize}

Oczekiwane wartości mogą zostać wyznaczone przez administratora organizacji oraz przez 
pracownika, który jest do danego pomieszczenia przypisany. W oczekiwaniach określane
są wartości temperatury oraz natężenia światła, a także dopuszczalne odchylenia od
pożądanych wartości. Taki zbiór informacji określany jest jako reguła.

Dla każdego z pomieszczeń można określić wiele reguł, z których każda będzie obowiązywać
w innym okresie w ciągu tygodnia. Przykładowo, dobrą praktyką byłoby określenie
odrębnej reguły obowiązującej w ciągu godzin pracy, gdzie oczekiwana wartość natężenia
światła wynosiłaby 600 lx oraz innej reguły, obowiązującej
poza godzinami pracy, gdzie oczekiwana wartość natężenia światła wynosiłaby 200 lx.
Po przekroczeniu tej granicy system poinformowałby użytkowników, że w danej sali
najprawdopodobniej pozostało zapalone światło. System automatycznie wykrywałby te i inne
nieprawidłowości, które można by następnie wyeliminować poprzez włączenie/wyłączenie światła
lub ogrzewania.

Wewnątrz systemu zostały zdefiniowane reguły domyślne obowiązujące tak długo, dopóki
użytkownicy nie utworzą swoich własnych. Wszystkie rodzaje reguł zostały przedstawione 
poniżej w kolejności obowiązywania:

\begin{itemize}
    \item domyślna
    \item organizacyjna
    \item oddziałowa
    \item spersonalizowana
\end{itemize}

Reguły organizacyjne mają pierwszeństwo przed regułami domyślnymi. Obowiązują dla
każdego oddziału i pomieszczeń należących do danej organizacji. Reguły oddziałowe
nadpisują reguły organizacyjne i obowiązują we wszystkich pomieszczeniach wewnątrz
danego wydziału. Reguły spersonalizowane posiadają największy priorytet i obowiązują
wyłącznie dla konkretnego pomieszczenia, dla którego zostały utworzone.

W kolejnych rozdziałach opisano zagadnienia związane z utworzeniem systemu. W rozdziale 2. 
przedstawiono różnice między już istniejącymi rozwiązaniami a proponowanym rozwiązaniem. 
Rozdział 3. został poświęcony założeniom, na których opierano się podczas tworzenia pracy. 
Treść rozdziału 4. zawiera porównanie dwóch konkurencyjnych architektur systemu oraz
argumenty przemawiające za wyborem architektury opartej na mikrousługach.
W rozdziale 5. przybliżono zagadanienia związane z modelowaniem danych oraz ich 
przechowywaniem. Rozdział 6. opisuje różne sposoby komunikacji między 
mikrousługami, wykorzystane w ramach pracy. Przygotowano zestaw testów sprawdzających
poprawność działania systemu, przedstawionych szczegółowo w rozdziale 7.
Treść rozdziału 8. zawiera szczegóły implementacyjne dotyczące dokonywania pomiarów
oraz przesyłania ich w taki sposób, aby wiadomości były zrozumiałe dla systemu.
Kolejny rozdział przybliża zasady działania oraz możliwości oferowane przez konteneryzację
systemu. Ostatni rozdział stanowi podsumowanie dokonań oraz prezentuje możliwe
dalsze kierunki rozwoju pracy.