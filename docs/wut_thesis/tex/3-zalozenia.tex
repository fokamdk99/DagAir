\newpage
\section{Założenia}
Funkcjonalność i architektura systemu została utworzona w oparciu o kilka istotnych 
założeń:

\begin{itemize}
    \item Szcególny nacisk jest położony na łatwość wdrożenia - system powinien być 
    gotowy do wdrożenia na środowisko 
    chmurowe. Organizacja zainteresowana uruchomieniem aplikacji dla swoich potrzeb 
    może wybrać opcję, w której dostarczane są obrazy odpowiednich serwisów oraz 
    skrypty konfigurujące środowisko. Takie rozwiązanie mogłoby być ofertą skierowaną 
    do banków, które chcą zminimalizować ruch zewnętrzny. Może także skorzystać z 
    opcji, w której system jest hostowany na serwerach firmy będącej autorem 
    oprogramowania
    \item System składa się z czujników zbierających pomiary, które następnie przesyłane 
    są do serwisów, które je przetwarzają. Do pomiaru zalicza się aktualna 
    temperatura, natężenie światła oraz jakość powietrza
    \item Aplikacja oparta jest na regułach określających oczekiwaną wartość powyższych 
    parametrów w danej chwili czasu. Po otrzymaniu każdego z pomiarów porównywane są 
    wartości oczekiwane z rzeczywistymi i na tej podstawie aplikacja przygotowuje wynik. 
    Domyślnie istnieje reguła podstawowa, gdzie oczekiwana temperatura wynosi 24,50. 
    Więcej informacji odnośnie tego skąd taka wartość została ustalona można 
    uzyskać, patrząc na tabelę 1
    \item System przewiduje dwie role użytkowników: pracowników danej 
    organizacji, którzy mogą tworzyć własne reguły dla pomieszczeń do nich 
    przypisanych, oraz administratorów organizacji, którzy posiadają wszystkie 
    uprawnienia przypisane pracownikom, a ponadto możliwość zarządzania informacjami 
    dotyczącymi organizacji, budynków, pomieszczeń i czujników
    \item System jest rozwijany przy wykorzystaniu platformy .NET w wersji 5.0
\end{itemize}

Tabela \ref{tab:optymalna-temperatura} pokazuje porównanie wyników z różnych artykułów traktujących 
o optymalnej temperaturze w pomieszczeniach:

\begin{xltabular}{1\textwidth}
    { |c|c| }
    \caption{Porównanie wyników badań estymujących optymalną temperaturę} \label{tab:optymalna-temperatura} \\
     \hline
     Badanie & Optymalna temperatura \\ 
     \hline
     \cite{Lan2012} & 23.5\degree C - 25.5\degree C \\ 
     \hline
     \cite{dai2014} & 23.0\degree C - 26.5\degree C \\ 
     \hline
     \cite{hedge2005} & 24.0\degree C - 25.0\degree C \\ 
     \hline
\end{xltabular}

W oparciu o powyższe badania wyliczono średnią optymalną temperaturę wynoszącą 24.5\degree C.
Natomiast w tabeli \ref{tab:optymalne-natezenie} porównywane są rezultaty badań nad optymalnym 
poziomem natężenia światła, który skutkował najlepszą efektywnością pracowników:

\begin{xltabular}{1\textwidth}
    { |c|c| }
    \caption{Porównanie wyników badań estymujących optymalne natężenie światła} \label{tab:optymalne-natezenie} \\
     \hline
     Badanie & Optymalne natężenie światła \\ 
     \hline
     \parencite{chinchiuan2014} & 500 lx \\ 
     \hline
     \parencite{liu2017} & 600 lx \\ 
     \hline
\end{xltabular}

W oparciu o powyższe badania wyliczono średnią optymalne natężenie światła 
wynoszące 550 lx.