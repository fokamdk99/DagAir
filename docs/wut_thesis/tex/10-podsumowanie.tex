\clearpage
\section{Podsumowanie}

W ramach niniejszej pracy udało się osiągnąć następujące rezultaty:

\begin{itemize}
    \item Każdy z serwisów został utworzony, uruchamia się prawidłowo i skutecznie
    komunikuje się z innymi serwisami przy pomocy brokera wiadomości lub
    żądania http
    \item Utworzono obrazy kontenerów każdego z serwisów. Uruchamiają się prawidłowo
    i skutecznie komunikują się między sobą, zarówno na środowisku lokalnym, jak
    i wewnątrz klastra kubernetesowego
    \item Graficzny interfejs użytkownika pozwala zarządzać organizacją, oddziałami
    oraz pomieszczeniami, wyświetla także aktualne wyniki pomiarów
    \item Zestaw pomiarowy w regularnych odstępach czasu zbiera pomiary i wysyła je 
    na kolejkę do brokera wiadomości. Wiadomości są pobierane przez SSDS i przesyłane 
    dalej
    \item Skonfigurowano środowisko na platformie Azure, dzięki czemu system
    był publicznie dostępny w internecie. Ze względu na znaczące koszty utrzymania 
    środowiska zdecydowano się na jego usunięcie. Możliwe jest jednak
    szybkie przywrócenie do pełnej funkcjonalności.
\end{itemize}

System do zarządzania warunkami technicznymi w pomieszczeniach biurowych jest publicznie
dostępny w serwisie GitHub\cite{github2022}.

\newpage
\section{Ograniczenia}

Pierwszym z możliwych kierunków dalszego rozwoju jest dodanie następujących komponentów:

\begin{itemize}
    \item Rozszerzenie styków udostępnianych przez mikroserwisy o nowe usługi. W rozdziale
    \ref{section:komunikacja-miedzy-serwisami}. przedstawiono listę zaimplementowanych 
    styków. Nie definiują one m. in. usług, które umożliwiały by aktualizację przechowywanych
    danych i akceptujących czasownik HTTP UPDATE.
    \item Rozszerzenie interfejsu graficznego o nowe strony, dodanie przycisków umożliwiających
    wykorzystanie usług akceptujących czasownik HTTP UPDATE
\end{itemize}

Projekt został przygotowany z myślą o tym, by można było możliwie łatwo tworzyć nowe 
serwisy i integrować je z już istniejącymi. Ważnym elementem ułatwiającym dodawanie 
nowych usług jest jasne zdefiniowane styków oferowanych przez inne serwisy usługowe. 

Na ten moment nie zaimplementowano rozwiązań automatyzujących wykonywanie wymaganych 
czynności w przypadku, gdy warunki rzeczywiste panujące w danym pomieszczeniu nie 
spełniają oczekiwań. Jednym z kierunków dalszego rozwoju projektu jest wykorzystanie 
towarów produkowanych przez firmę Ikea. Zastosowanie inteligentnego oświetlenia 
wykorzystującego protokół ZigBee pozwoliłoby na automatyczne sterowanie poziomem 
natężenia światła przez aplikację. W tym celu należałoby utworzyć nowy serwis 
wykorzystujący gotową bibliotekę \textit{pytradfri} \cite{ikea2022}
pozwalającą na zarządzanie oświetleniem.

Wartą rozważenia opcją jest utworzenie dedykowanej aplikacji mobilnej na urządzenia
obsługujące systemy Android lub iOS. Aplikacja oferowałaby podobny zestaw funkcjonalności
co aplikacje webowe, a dodatkowo posiadałaby system powiadomień, który generowałby
wiadomości typu push za każdym razem przy wystąpieniu konkretnego zdarzenia, jak na 
przykład przekroczenie ustalonego progu temperatury lub natężenia oświetlenia.

Propozycją dalszego rozwoju projektu jest dodanie innych czujników i przeprowadzanie
kompleksowych pomiarów dla danego pomieszczenia. Możliwymi rozszerzeniami są między
innymi czujnik wilgotności powietrza lub czujnik jakości powietrza. Na ich podstawie
można by generować bardziej szczegółówe wyniki i w ten sposób jeszcze bardziej
zwiększyć wydajność pracowników oraz ograniczyć zużycie energii. Atutem utworzonego
systemu jest łatwość dodawania nowych czujników, wystarczy je bowiem podpiąć 
przy pomocy zestawu przewodów połączeniowych do wolnych wejść mikrokontrolera, a następnie dodać
oprogramowanie umożliwiające odczyt pomiarów. 

Do tej pory wystarczającym instrumentem do sprawdzenia poprawności działania systemu
było przeprowadzenie testów jednostkowych, integracyjnych oraz end-2-end. Jednak
wraz z dalszym rozwojem systemu konieczne będzie skonfigurowanie jego monitoringu.
Powodów jest wiele, między innymi:

\begin{itemize}
    \item Monitorowanie ogólnego stanu, w jakim znajduje się system
    \item Zbieranie metryk umożliwiających oszacowanie wydajności systemu
    \item Upewnienie się, że oferowane przez mikroserwisy usługi są dostępne i na bieżąco przetwarzają żądania
    \item Możliwość odnalezienia fragmentów systemu, które spowalniają jego działanie
    lub w ogóle nie funkcjonują
\end{itemize}

Jednym z istniejących narzędzi do zbierania kolekcjonowania metryk opisujących
stan systemu jest Prometheus\cite{prometheus2022}. Jest to narzędzie do monitorowania
aplikacji, w szczególności przeznaczone do systemów rozproszonych, opartych na
architekturze mikroserwisowej. Pozwala agregować dane dotyczące zużycia procesora,
pamięci czy dysku, a także tworzyć własne metryki, które mogą przykładowo mierzyć
czas potrzebny na przetworzenie przychodzących żądań http.