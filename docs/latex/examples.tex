\documentclass[11pt, a4]{article}
\usepackage[utf8]{inputenc}
\title{Praca inzynierska}
\author{Stanislaw Skrzypek}
\date{January 2022}
\usepackage{graphicx}
\graphicspath{ {images} }

\begin{document}
\maketitle

\begin{abstract}
    This is a simple paragraph at the beginning of the 
    document. A brief introduction about the main subject.
\end{abstract}

\tableofcontents

\section{Introduction}

This is the first section.

Lorem  ipsum  dolor  sit  amet,  consectetuer  adipiscing  
elit.   Etiam  lobortisfacilisis sem.  Nullam nec mi et 
neque pharetra sollicitudin.  Praesent imperdietmi nec ante. 
Donec ullamcorper, felis non sodales...

asdasd

\section{Second Section}

Lorem ipsum dolor sit amet, consectetuer adipiscing elit.  
Etiam lobortis facilisissem.  Nullam nec mi et neque pharetra 
sollicitudin.  Praesent imperdiet mi necante...

\subsection{First Subsection}
Praesent imperdietmi nec ante. Donec ullamcorper, felis non sodales...

\subsubsection{Third subsubsection}
Nullam nec mi et neque pharetra 
sollicitudin.  Praesent imperdiet mi necante...

\section*{Unnumbered Section}
Lorem ipsum dolor sit amet, consectetuer adipiscing elit.  
Etiam lobortis facilisissem

Siemano!. This is a simple example, with no 
extra parameters or packages included.

% This line here is a comment. It will not be printed in the document.
1Some of the \textbf{greatest} % pogrubienie
discoveries in \underline{science} % podkreślenie
were made by \textbf{\textit{accident}}. % pogrubienie i kursywa

2Some of the greatest \emph{discoveries} % inside normal text the emphasized text is italicized, but this behaviour is reversed if used inside an italicized text
in science 
were made by accident.

\textit{3Some of the greatest \emph{discoveries} 
in science 
were made by accident.}

\begin{figure}[h] % dodanie obrazka
    \centering % wycentrowanie
    \includegraphics[width=0.75\textwidth]{thesis} % dodanie grafiki "thesis.jpg"
    \caption{a nice plot} % opis obrazka
    \label{fig:thesis1} % labelka do figury, dzieki ktorej mozna sie pozniej do obrazka odnosic
\end{figure}

As you can see in the figure \ref{fig:thesis1}, the %odniesienie do figury
function grows near 0. Also, in the page \pageref{fig:thesis1} % generacja strony na ktorej znajduje sie figura
is the same example.

\begin{itemize} % lista nienumerowana
    \item The individual entries are indicated with a black dot, a so-called bullet.
    \item The text in the entries may be of any length.
\end{itemize}

\begin{enumerate} % lista numerowana
    \item This is the first entry in our list
    \item The list numbers increase with each entry we add
\end{enumerate}

\begin{center}
    \begin{tabular}{ l c r } % left, center, right allignment
     cell1 & cell2 & cell3 \\ 
     cell4 & cell5 & cell6 \\  
     cell7 & cell8 & cell9    
    \end{tabular}
\end{center}

\begin{center}
    \begin{tabular}{ |c|c|c| } 
     \hline % horizontal line
     cell1 & cell2 & cell3 \\ 
     cell4 & cell5 & cell6 \\ 
     cell7 & cell8 & cell9 \\ 
     \hline
    \end{tabular}
    \end{center}

    Table \ref{table:data} is an example of referenced \LaTeX{} elements.

\begin{table}[h!]
\centering
\begin{tabular}{|c c c c|} 
 \hline
 Col1 & Col2 & Col2 & Col3 \\ [0.5ex] 
 \hline\hline
 1 & 6 & 87837 & 787 \\ 
 2 & 7 & 78 & 5415 \\
 3 & 545 & 778 & 7507 \\
 4 & 545 & 18744 & 7560 \\
 5 & 88 & 788 & 6344 \\ [1ex] 
 \hline
\end{tabular}
\caption{Table to test captions and labels}
\label{table:data}
\end{table}

\begin{table}[!ht]
    \caption{Porównanie popularnych architektur systemów}
    \label{tab:porownanie-architektur}
    \begin{tabularx}{1\textwidth} { 
        | >{\raggedright\arraybackslash}X 
        | >{\centering\arraybackslash}X 
        | >{\raggedleft\arraybackslash}X | }
        \hline
       Cecha & System monolityczny & System oparty na architekturze mikroserwisowej \\
       \hline
    \end{tabularx}
\end{table}

\end{document}