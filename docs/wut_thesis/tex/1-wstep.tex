\newpage % Rozdziały zaczynamy od nowej strony.
\section{Temat pracy}

Tematem niniejszej pracy jest utworzona w ramach seminarium dyplomowego aplikacja do 
zarządzania warunkami technicznymi w pomieszczeniach biurowych oparta na architekturze 
mikrousługowej. Produkt ma na celu poprawę warunków panujących w pomieszczeniach 
przeznaczonych do pracy codziennej. Wybrane parametry przeznaczone do optymalizacji to 
temperatura oraz natężenie światła.

Implementacja projektu przewiduje umieszczenie w badanych pomieszczeniach odpowiedniego 
rodzaju czujników, które będą na bieżąco monitorować stan danej przestrzeni. 
Zintegrowany z czujnikami system informacyjny powinien odczytywać przesyłane 
pomiary, a następnie je interpretować. Wynik interpretacji powinien być widoczny dla 
zainteresowanych osób. W wiadomości będą znajdować się informacje dotyczące 
akcji, które należy podjąć, aby umożliwić ustalenie się badanych parametrów na 
właściwym poziomie.

Utworzenie aplikacji miało przyczynić się do osiągnięcia dwóch głównych 
celów, stawianych od początku przygotowywania pracy:

\begin{enumerate}
    \item Poprawa warunków pracy - badania \cite{oseland2012} dowodzą, że 
    warunki panujące w pomieszczeniach do pracy mają wpływ na efektywność pracowników. 
    Utrzymanie ich na optymalnym poziomie może spowodować wzrost wydajności do 2,5\%
    \item Redukcja zużywanej energii - w praktyce często zdarza się, że po zakończeniu 
    pracy zostawiane są włączone światła na całą noc. Innym przykładem może być 
    sytuacja, w której pomieszczenie jest ogrzewane, mimo iż nikt z niego nie korzysta. 
    Przy wsparciu aplikacji będzie możliwe zapobieganie takim wydarzeniom, co w 
    konsekwencji ograniczy zużycie energii
\end{enumerate}

Zgodnie z wynikami badań opublikowanymi przez Institute for market transformation z 
2015 roku około 40\% całkowitej konsumpcji energii przypada na zasilanie budynków 
\cite{Imt.org2015}. Przekłada się to w ciągu roku na wydatek rzędu 450 miliardów 
dolarów. Najsłabiej zagospodarowane budowle zużywały od trzech do siedmiu razy więcej 
energii od tych najbardziej oszczędnych. Istnieje zatem potrzeba przygotowania 
i wdrożenia rozwiązań, które z jednej strony nie byłyby obciążające finansowo, z 
drugiej strony zaś ograniczające już istniejące koszty. 