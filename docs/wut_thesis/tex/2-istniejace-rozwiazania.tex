\newpage
\section{Istniejące rozwiązania}
Firma Sharp przygotowała podobne rozwiązanie, za pomocą którego można mierzyć kluczowe 
parametry danego pomieszczenia, przesyłać je na platformy chmurowe i je analizować 
\cite{sharp2022}. Różnica między tym produktem a rozwiązaniem proponowanym w tej pracy 
polega na tym, że w rozwiązaniu firmy Sharp czujniki są wbudowane w monitor służący 
jako centrum telekonferencyjne. W ten sposób wykonywane pomiary stają się niejako 
dodatkiem do monitora, niż głównym celem wstawienia urządzenia do konkretnej sali. 
W konsekwencji, wykonywanie pomiarów w wielu salach wiązałoby się z koniecznością 
zakupu drogiego monitora dla każdej z nich. Proponowane w tej pracy rozwiązanie zawiera 
jedynie zestaw czujników przesyłających pomiary do systemu, bez innych dodatków, co 
znacznie minimalizuje koszt wdrożenia takiego rozwiązania. Dzięki temu opłacalne staje
się zbieranie pomiarów z wielu sal jednocześnie.